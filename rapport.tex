\documentclass[a4paper, 12pt]{article}

\input{headers} % A priori, vous n'aurez pas besoin de modifier le contenu de ce fichier :)

\begin{document}
\pagenumbering{roman} 

%%% Remarque sur la page de garde :
% Il existe une commande beaucoup plus simple : \maketitle
% Cette commande ne permet pas directement l'insertion de logo et de données 
% additionnelles sans modifier certains fichiers de configuration
% (utilisation avancée)

\begin{titlepage}
\setlength{\headheight}{0cm}
\setlength{\headsep}{0cm}
{

%%% Insertion des logos [begin]
\makelogos
%%% Insertion des logos [end]

\vspace{4cm}

\begin{center}
\fbox{ 
\begin{minipage}[h]{.9\linewidth}
\begin{center}
{\vspace*{5mm}
\huge\textbf{Rapport d'Ouverture Scientifique et Technique}\\  %%% Titre du rapport
\vspace*{5mm}}
\end{center}
\end{minipage}
}

\vspace{15mm}

Auteur(s)\\~\\
{\large 
\bsc{DEMARS} Thomas\\
\bsc{JENNATE} Ibtissam\\
\bsc{SANSANÉ} Hugo}\\
~\\
\underline{STI, 4A}\\ 

\vspace{3cm}  

\textbf{Année Universitaire 2023 - 2024\\
{\tiny version : \today}}

\vspace{2cm}  

\end{center}
  
\vfill

\begin{flushleft}
	Encadrant : \textsc{Ciucanu} Radu
\end{flushleft}

}
\end{titlepage}

\newpage		
\tableofcontents % Insertion de la table des matières
\addcontentsline{toc}{section}{Table des matières}

% Vous pouvez également pour des rapports plus longs (des rapports de stages par exemple) insérer une table des figures
%\listoffigures
%\addcontentsline{toc}{section}{Liste des figures} 

% Voir même une liste des algorithmes
%\listofalgorithms
%\addcontentsline{toc}{section}{Liste des algorithmes}

\clearpage 

\pagenumbering{arabic} 
\section{Introduction / Contexte}

\paragraph{}
L'article « Scalable Linear Algebra on a Relational Database System »\cite{10.1145/3277006.3277013} a été publié en 2017 par les auteurs \textit{Shangyu Luo, Zekai J. Gao, Michael Gubanov, Luis L. Perez, Christopher Jermaine}. Il aborde un sujet crucial dans le domaine de la gestion des données et de l'analyse : l'intégration des opérations de calcul matriciel dans les bases de données relationnelles. Les auteurs ont proposé des méthodes novatrices pour améliorer les performances des bases de données relationnelles, ouvrant de nouvelles perspectives pour l'analyse des données à grande échelle.

Cet article a été publié en 2017, une époque où les entreprises et institutions recueillaient d'énormes quantités de données, venant d'une multitude de sources telles que les réseaux sociaux, le commerce en ligne et les appareils liés à l'Internet des Objets (IOT). Toutes ces données étaient déjà considérées à l'époque, comme une mine d'or d'informations importantes à exploiter pour les entreprises et les institutions. Mais, les bases de données relationnelles, telles qu'à l'époque, n'étaient pas du tout optimiser pour les calculs matriciels, qui sont au centre dans des domaines tels que la simulation, l'analyse de données ou bien encore l'apprentissage automatique.

\clearpage 
\section{Problématique}

\paragraph{}
Comme vu précédemment, les bases de données n'étaient pas bien adaptées à l'exécution rapide d'opérations de calcul matriciel. Au lieu de cela, ces bases de données étaient principalement conçues pour des opérations de requête SQL standard, telles que le choix, la mise à jour et la suppression de données.
 
La question abordée par les créateurs est donc la suivante : 

\begin{center}
	\textbf{
		Comment surmonter les limites des bases de données relationnelles pour permettre l'exécution efficace d'opérations de calcul matriciel sur de grands ensembles de données ?
	}
\end{center}

\clearpage 
\section{Apports scientifiques principaux de l’article}

Les auteurs de cet article ont présenté plusieurs avancées dans le domaine de la gestion et de l'analyse des données, avec une idée novatrice, implémenter le calcul matriciel dans les bases de données relationnelles. Voici un aperçu simple mais détaillé de ces avancées :

\subsection{Intégration des opérations matricielles}
Les bases de données relationnelles traditionnelles n'ont pas été créées pour supporter efficacement des opérations de calcul matriciel, ce qui peut limiter leur utilité lors de traitements de grandes quantités de données. Les auteurs ont proposé des méthodes innovantes pour intégrer directement des opérations matricielles dans le moteur de traitement des requêtes de la base de données, ce qui permet aux utilisateurs d'effectuer ces opérations complexes à l'aide du langage SQL standard, sans avoir besoin d'outils ou de langages externes.

\subsection{Optimisation des performances}
La majorité de cet article se concentre sur l'optimisation des performances. Les auteurs ont créé des concepts et développé des techniques afin de rendre plus efficace le stockage des données et d'organiser les opérations matricielles afin de réduire au maximum le temps nécessaire d'exécution lors de l'utilisation de ces opérations. Ces optimisations ont considérablement amélioré la vitesse et l'efficacité des calculs matriciels, même sur de grands ensembles de données.

\subsection{Évolutivité}
L’évolutivité, au sens des bases de données, est la capacité d'un système à gérer la croissance du volume et la diversité des données ou requêtes sans perdre de performance. Savoir adapté l'architecture données est un enjeu crucial et les auteurs ont pu se pencher sur comment distribuer des calculs, notamment matriciels, en parallèle (simultané), ce qui utilise les capacités de traitement réparti des systèmes de base de données actuels. Cela a permis de gérer de grandes quantités de données tout en maintenant d'excellente performances.

\paragraph{}
Grâce à ces avancées, les auteurs ont pu transformer les bases de données relationnelles en un outil performant et polyvalent dans l'analyse avancée de données de masse. Cela rend la possibilité d'utilisation de ce système dans des domaines où les opérations matricielles sont déjà présentes. Toutes ces avancées n'ont pas eu qu'un impact sur la gestion des données, mais aussi sur la manière dont les chercheurs et académiques abordent l'analyse de données à grande échelle, ouvrant l'opportunité pour encore plus d'innovations et de découvertes.

\clearpage 
\section{Impacts de l'article}

Nous allons maintenant étudier l’impact de cet article sur 2 aspects différents. 

\subsection{La recherche universitaire}

Cet article a eu un impact significatif dans le domaine de la recherche sur la gestion et l'analyse de données massives. Grâce à l'introduction de méthodes innovantes pour l'utilisation et l'optimisation d'opérations matricielles dans les systèmes de bases de données relationnelles, les auteurs ont ouverts un nouveau domaine pour les chercheurs dans la gestion de données.

Les avancées proposées dans cet article ont créé un intérêt massif dans la communauté universitaire, stimulant de nouvelles études et de nouveaux travaux dans le but d'améliorer les approches de cet article. Nous avons par exemple, « Expressive Power of Linear Algebra Query Languages » \cite{10.1145/3452021.3458314} qui se base sur les avancées de l’article ou encore « Distributed numerical and machine learning computations via two-phase execution of aggregated join trees »\cite{10.14778/3450980.3450991} qui utilise les concepts de l’article pour optimiser les méthodes avancées d’analyses de données.

\subsection{Domaines applicatifs}

Les innovations présentées dans l'article ont un impact majeur dans divers domaines d'application. En voici quelques exemples : 


\begin{enumerate}[label={-}]
	\item \textbf{Finance :} Dans la finance, le traitement matriciel sur les systèmes de base de données relationnelles a permis d'améliorer la performance des modélisations des risques financiers, la détection des fraudes, l'optimisation de portfolio d'investissement et la prédiction de tendance du marché. Les entreprises dans la finance ont ainsi pu prendre des meilleures décisions et ont pu gérer plus efficacement les risques associés au domaine.
	\item \textbf{Soins de santé :} Dans le domaine des soins de santé, les techniques de l'article ont permis d'analyser d'immenses quantités de données médicales, cela inclut les dossiers des patients, les résultats d'analyses et les données génomiques (étude du fonctionnement d'un organisme, d'un organe, d'un cancer, etc..). Cela a comme impact l'amélioration des suivis, la personnalisation de traitements, la gestion des ressources médicales ou encore la détection précoce de maladies.
	\item \textbf{Marketing :} Les entreprises dans le marketing utilisent le calcul matriciel sur des bases de données pour analyser les données de ces clients, les comportements d'achat ou bien encore les comportements sur les réseaux sociaux. Cela permet d'augmenter l'efficacité de campagnes marketing, d'encore mieux personnalisation les offres et les recommandations de produits et d'optimiser les stratégies de tarification.
\end{enumerate}


\clearpage 
\section{Analyse critique du travail proposé}

\paragraph{}
L'innovation apportée par l'intégration des calculs matriciels dans les bases de données relationnelles soulève des perspectives enthousiasmantes pour la gestion et l'analyse des données à grande échelle. Cependant, cette avancée technologique n'est pas exempte de questionnements éthiques, particulièrement en ce qui concerne la protection de la vie privée et l'usage des données. La capacité d'effectuer des opérations complexes sur de vastes ensembles de données augmente inévitablement le risque de manipulations inappropriées ou non autorisées des informations sensibles.

\paragraph{}
Il est donc primordial que les avancées technologiques soient accompagnées d'un cadre éthique robuste, qui veille non seulement à la sécurisation des données mais également à leur utilisation équitable et transparente. La responsabilité incombe aux chercheurs et aux développeurs de garantir que les technologies ne soient pas seulement performantes, mais aussi respectueuses des normes éthiques, en implémentant des mécanismes de contrôle stricts et en promouvant une utilisation consciente et responsable des données.
Intégrité du protocole de recherche

\paragraph{}
L'article montre une approche rigoureuse et méthodique, depuis la conception théorique jusqu'à l'expérimentation pratique, révélant ainsi une intégrité notable du protocole de recherche. L'introduction de modifications au sein des systèmes de gestion de bases de données pour accommoder les calculs matriciels est soutenue par une argumentation claire et des résultats expérimentaux probants. Cette transparence méthodologique favorise la compréhension et la réplicabilité des résultats, constituant un fondement solide pour des recherches futures.

\paragraph{}
Néanmoins, l'analyse pourrait gagner en profondeur en abordant de manière plus exhaustive les contraintes et les défis liés à la mise en œuvre de cette technologie. Bien que l'article mette en lumière les bénéfices significatifs de l'intégration des opérations matricielles dans les bases de données, une réflexion critique sur les barrières techniques, la transition pour les systèmes existants et l'impact sur les utilisateurs finaux enrichirait la discussion. Comprendre ces aspects permettrait de mieux cerner le potentiel de généralisation de cette innovation et d'identifier les axes nécessaires pour une adoption plus large et plus efficace.

\clearpage 
\section*{Conclusion}
\addcontentsline{toc}{section}{Conclusion}

\paragraph{}
Ce sujet nous a permis de découvrir le monde de la recherche sur plusieurs aspects :

\paragraph{}
Tout d'abord, il nous a permis de nous familiariser avec les outils latex lors de la rédaction de ce rapport, bien que les éléments les plus complexes aient été déjà mis en place dans le template fournis. Cette découverte a été relativement positive et certains membres de l’équipe envisagent de l’utiliser dans la rédaction de futurs rapports.

\paragraph{}
De manière plus indirecte, ce projet nous a permis de nous familiariser avec les différentes étapes du début de la rédaction d’un papier de recherche, recherches sur l’état de l’art, lecture de papiers de recherche sur plusieurs niveaux (rapide dans un premier temps afin de déterminer si l’article est intéressant pour notre sujet, puis approfondis dans le but de mieux le comprendre).

\paragraph{}
De plus, ce projet nous a permis de développer un esprit critique de notre lecture et dans la rédaction tout au long de l’article, pour cela, il a également été nécessaire d’en savoir plus sur le fonctionnement des bases de données au-delà de la description fait dans l’article.


\clearpage 
\bibliographystyle{plain}
\bibliography{bibliographie}
\addcontentsline{toc}{section}{Références}


% \clearpage 
% \appendix
% \bigskip\noindent{\Large\bf Annexes}
% \addcontentsline{toc}{section}{Annexes}
% \section{Algorithme qui fait quelque chose}


% \clearpage 
% \section{Une autre annexe}

\end{document}